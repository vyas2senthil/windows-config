% This is the file GB.tex of the CJK package
%   for testing Chinese (in GB encoding).
%
% written by Werner Lemberg <wl@gnu.org>
%
% Version 4.8.0 (22-May-2008)

\documentclass[10pt]{article}

\usepackage{CJK}
\usepackage{parskip}

\begin{document}
\begin{CJK*}{UTF8}{simsun}
\CJKtilde

\title{程序员的矛盾论}
\date{}
\author{包昊军}
\maketitle

\section{}

矛盾无处不在。矛盾就是江湖。只要有人的地方,就有江湖。只要有人的地方,
也就有矛盾。做为一个程序员,很不幸,我们也是人。而不是神。只有Linus那样
的人才敢公然宣称自己是神。别人没了他不行。结果还是让很多人不舒服了。

程序员是干什么的?写程序的。程序本来没有,程序员把它写出来,程序就有了。
有和没有,是最经典的矛盾。就像0和1。

程序员写出来的程序,有的很好,有的很烂,矛盾。

程序员写程序要用编程语言,编程语言有很多种,有C,C++,选C还是C++呢?矛
盾。还有Java,Python,Perl,Bash,Sed,Awk,Lisp,Ruby,汇编。纸带打孔。

程序员都需要花时间学习,写程序也要花时间。学得太多了,会导致写程序的时
间少了,相应的写的程序也少了,于是水平可能就不会有写过很多程序的人高。

用什么操作系统呢?Windows? 还是Linux?

用什么操作界面呢?图形的?命令行的?

用什么编辑器呢?Vim?Emacs?

用集成开发环境?不用?

矛盾无处不在,程序员写的每一行代码,都是矛盾冲突的结果。

再说说非技术方面的矛盾。程序员跟周围的同事,处于一种合作竞争的关系,其
中无疑也充满着矛盾。

程序员跟老板,也充满着矛盾。老板管理程序员,程序员也需要管理老板。有的
老板很聪明,却装作很笨;有的老板很蠢,却爱抖聪明。

程序员一般都受雇于人,作为雇员,跟雇主之间也充满矛盾。尤其是当雇主降工
资的时候。

更不要说程序员面对问题、deadline、问责等等等等的时候的压力与动力之间的
矛盾。

这么说下去,再说个三天三夜也说不完矛盾。所以还是让我们切入正题吧。

\section{面对这么多矛盾,怎么办}

这个问题,是没有答案的。这是一个哲学问题,而所有的哲学问题,都没有答案。
就像生与死,作为一个人,每一个人都得面对。有些人选择不活了,有些人选择
快乐地活着,还有些人选择堕落,有些人选择活得不明不白的。哦错了,最后一
种人不作选择。

当一个快乐的程序员,其实很简单。

\section{学习}

程序员一生都应该不断学习。这一点是毫无疑问的。与学习相对立的,不是不学
习,而是创作。创作出东西来让别人去学习。不学习不创作的时候,大家爱干嘛
干嘛,咱们这里就不说了;创作是下一节会讲的。所以这一节主要讲讲学习。

怎么学习真的很重要。一个人的话大致有三种学习的方法:读书,读代码,读文
档。

\subsection{读书}

书,好像谁都会读,因为谁都是十几年甚至二十年的书读下来的。我也一直以
为,自己对读书应该没有任何问题,no problem at all。

直到我读了那本书,我才知道,问题出大发了。

那本书的名字叫《How to read a book》。

二十多年来,我从来没有想过,自己为什么要读书;从来不知道,一本书,除了
从头读到尾之外,还有其他不同的读法。读了那本书之后,我全知道了。

读一本书,以前我一直以为是为了考试能拿高分。所以当后来我没有试考了的时
候,我突然觉得自己不会看书了。真是被应试教育给害惨了。

看书千万不要每一本都从头看到尾。二十一世纪最缺的是什么?人才!人才最缺
的又是什么呢?时间!作为一个高科技人才,我们真的不能那么奢侈。

所以如果你决定一本书你要从头读到尾的话,请确保这是你的一个理智的,明智
的,清醒的决定,而不只是你二十多年的习惯在作祟。

怎么作出理智、明智、清醒的决定呢?

首先要克服自己的那个二十多年的习惯。这不是一件容易的事。我们必须认识到
了,到现在这把年纪了,不是每本书都值得那样去读。一个娃娃拿到一本书,里
面的东西它啥啥都不懂,所以必须从头到尾的读。所以国家教育机构要做好选教
材的工作,尽量保证娃娃们读到的书不是坏书。不值得读的书。

如果作为一个成年人,还不会自己选书,不对自己读什么书怎么读负责任的话,
那就还是停留在娃娃时代,或者说的难听一点是脑残了。娃娃们有的是时间,成
年人没有。娃娃们的世界很简单,没有那么多“矛盾”,成年人很遗憾,很复杂,
充满矛盾。

\subsection{快速阅读}

读书一定要快,时间就是生命。一本书看不懂,扔掉,换一本。把所有书都看完
了怎么办?把以前扔掉的再捡起来,看看现在是不是好一点,能多看懂一点了。
这是最简单,最不动脑子的做法了。

后来,看了那本书之后,我又学会了一些新的方法。

比如,尽量理解作者,学习作者的语言。不要奇怪,作者说的难道不就是中文或
者英文吗?是的,但又不是。每个人说的语言都是不一样的,都是一种独特的,
只属于说话人一个人的语言。就像人的掌纹、瞳孔、声音有不同,语言也有特征。
红楼梦研究里一项很重要的发现是一个搞计算机的人作出来的,他用统计学原理
对前八十回和后四十回的语言进行了分析对比,结果得出来这两部分好像的确不
是曹雪芹一个人写的。

所以当你读比如一本中文书的时候,你以为那全是你的母语,结果你就放松警
惕,认为不需要费什么功夫就能读懂,结果不料的是,这其实是一门外语。你应
该把它当成一本外语书来读,尽量理解作者的语言;表达一个意思可以用不同的
词汇,他的词汇集可能跟你的有差异,你应该尽快消除这种差异,诸如此类。我
说的没有那本书好,所以推荐你尽快去读一下那本书吧。

最重要的还是你自己要战胜自己的想把书从头读到尾的“本能”,二十多年刻苦训
练出来的本能,现在你应该做一些反训练,比如,故意的胡乱、飞快地读一本书。

我想从头读到尾,可是我没钱没地位。

\subsection{融会贯通}

书读多了以后自然而然的会发现,其实很多书里面都讲得是一个道理。但是又讲
得有点不一样。这时候你需要去做一些联合阅读。就是把几本书放到一起读,有
对比的读。还是推荐你去看那本书。

总而言之,就是一个字,快。干什么都得快,除了做爱。那也得要有快感哪不是。

\section{创作}

之前为讲学习而一直隐忍不说,其实我觉得压根就不应该学习,如果你不学习就
能搞创作的话。

\subsection{快速阅读}






\end{CJK*}

\end{document}

%%% Local Variables:
%%% coding: utf-8
%%% mode: latex
%%% TeX-master: t
%%% End:
